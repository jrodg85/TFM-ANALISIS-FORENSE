El entorno de trabajo para un análisis forense enfocado en la exploración de memoria RAM y disco duro exige una meticulosa preparación y adecuación de las herramientas y espacios de trabajo. Las evidencias, provenientes tanto de la RAM como del almacenamiento persistente del ordenador en cuestión, se convierten en el pilar fundamental del análisis, permitiendo la evaluación de procesos en ejecución, archivos almacenados, registros de actividad y cualquier otro elemento que pueda arrojar luz sobre las acciones realizadas en la máquina.

En un segundo plano, pero no menos esencial, se encuentra el portátil personal, que se configura como la estación de trabajo principal para la realización del análisis forense. Este debe estar equipado con un sistema operativo que, comúnmente en el ámbito forense, suele ser alguna distribución de Linux, junto con una serie de herramientas específicas para el análisis forense (como Autopsy o Sleuth Kit). No obstante, la selección y configuración de estas herramientas incurren en una deuda técnica que debe ser minuciosamente administrada, asegurando la pertinencia, licencia y compatibilidad de las mismas.

Relativo al ordenador personal destacar las siguientes aplicaciones que se van a utilizar para la realización del análisis.

{\color{red}\textbf{DEUDA TÉCNICA: Listado de aplicaciones a utilizar en la descripción del entorno de trabajo}}

Por otro lado, la documentación y redacción del Trabajo de Fin de Máster (TFM) se consolida mediante el uso del repositorio en GitHub TFM-ANÁLISIS-FORENSE (https://github.com/jrodg85/TFM-ANALISIS-FORENSE). Este repositorio no solo sirve como medio para documentar y presentar los hallazgos y metodologías empleadas, sino que también se erige como una herramienta para gestionar versiones y cambios a lo largo del desarrollo del trabajo, facilitando la trazabilidad y coherencia del mismo. Se deben establecer estrategias robustas para garantizar la integridad y confidencialidad de la información almacenada, considerando la naturaleza sensible de los datos manejados en la investigación forense.

Finalmente, Internet emerge como un recurso invaluable para la investigación, actualización y comunicación a lo largo del proyecto. Navegar por la red debe ser realizado de forma segura y consciente, protegiendo las comunicaciones y asegurando la integridad de las herramientas y datos descargados.