Por, lo expuesto en la introducción del capítulo, se coliga que el problema a resolver es la resolución de las cuestiones solicitadas por el Gerente de la empresa. 

Una definición idonea que se puede adoptar en el presente TFM es lo indicado en su momento en la propuesta del TFM:

Solventar las necesidades del gerente de la empresa mediante el análisis forense del disco duro y la captura de memoria de un ordenador personal, en un caso real con un sistema virtualizado, vinculado a una presunta conducta delictiva real. Para ello, se utilizarán herramientas específicas para la localización de las evidencias digitales sobre los discos duros y la mameoria que puedan demostrar el presunto delito (Encase, Autopsy, Volatility, o cualquier otra herramienta, o conjunto de herramientas con prestaciones equivalentes). Finalmente, las evidencias localizadas deberán recogerse en un informe ejecutivo o pericial, el cual, además de los aspectos técnicos, deberá tener en cuenta aquellos requisitos procesales necesarios para que el análisis pueda tener validez en un proceso judicial.

{\color{red}\textbf{DEUDA TÉCNICA: Pendiente de Referenciar!!!, referenciar de la web de los TFM}}