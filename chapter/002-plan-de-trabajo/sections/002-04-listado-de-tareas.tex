En esta sección se ha elaborado después de una planificación del trabajo, el cual se han designado el siguiente listado de tareas a realizar. Gracias a este listado, podemos organizar el cómo vamos a realizar el TFM

Destacar que durante el listado de las tareas, cabe mencionar que habrán tareas de grooming o refinamiento, ellas no son utilizadas para reducción de deuda técnica, el objetivo estas jornadas es reflexionar sobre el contenido del mismo y valorar posibilidad de mejorar la organización del mismo. Estas variaciones, gracias a que se está realizando un control de versiones con git, se podrán ver las evoluciones o cambios del TFM en el mismo.


\noindent Durante la elaboración del reto 1 (PEC 1), se realizarán las siguientes tareas.

\begin{enumerate}
    \item Lectura enunciado actividad 1.
    \item Decision de formato de TFM.
    \item Maquetación de TFM en LaTeX.
    \item Elaboración de índice.
    \item Refinamiento de TFM 1.
    \item Diagrama de Gantt.
    \item Problema a resolver.
    \item Objetivos.
    \item Revisión del estado del arte de la informática forense.
    \item Refinamiento de TFM 2.
\end{enumerate}



\noindent Durante la elaboración del reto 2 (PEC 2), se realizarán las siguientes tareas.

\begin{enumerate}
    \item Lectura enunciado actividad 2.
    \item Extremos de análisis y previsión de pruebas: Introducción.
    \item Extremos de análisis.
    \item Previsión de pruebas.
    \item Análisis de la memoria RAM: Introducción.
    \item Acciones previas al análisis de RAM.
    \item Búsqueda de procesos en funcionamiento.
    \item Análisis y extracción de procesos sospechosos.
    \item Listado de conexiones de red y conexiones sospechosas.
    \item Refinamiento TFM 3.
    \item Feedback de la PEC 01.
    \item Análisis de disco duro: Introducción.
    \item Acciones previas al análisis de disco duro.
    \item Datos de interés y usuarios del sistema del disco duro analizado.
    \item Análisis de las evidencias del disco duro.
    \item Planning relativo al resumen ejecutivo.
    \item Planning relativo al informe pericial.
    \item Adaptación al indice a los nuevos cambios en los capítulos 6 y 7.
    \item Refinamiento TFM 4.
\end{enumerate}



\noindent Durante la elaboración del reto 3  (PEC 3), se realizarán las siguientes tareas.

\begin{enumerate}
    \item Lectura enunciado actividad 3.
    \item Introducción Resumen ejecutivo.
    \item Análisis Ejecutivo.
    \item Conclusión de análisis ejecutivo.
    \item Refinamiento TFM 5.
    \item Feedback de la PEC 02.
    \item Introducción del informe pericial.
    \item Cuerpo del informe pericial.
    \item Conclusiones del informe pericial.
    \item Conclusiones TFM.
    \item Revision de terminos abrebiaturas y acrónimos.
    \item Revisión de imágenes.
    \item Revision de referencias.
    \item Refinamiento TFM 6.
\end{enumerate}


\noindent Durante la elaboracion del reto 4  (PEC 4), se realizarán las siguientes tareas.

\begin{enumerate}
    \item Revisión de las anotaciones y consejos de la tutora de TFM 1.
    \item Ultimas correcciones Feedback TFM 1.
    \item Revisión de las anotaciones y consejos de la tutora de TFM 2.
    \item Ultimas correcciones Feedback TFM 2.
\end{enumerate}

\noindent La Entrega de videos, presentacion y realización de la defensa del TFM, se consideran que estan fuera de este TFM, ya que a partir de la fecha se considera entregado el presente documento.
