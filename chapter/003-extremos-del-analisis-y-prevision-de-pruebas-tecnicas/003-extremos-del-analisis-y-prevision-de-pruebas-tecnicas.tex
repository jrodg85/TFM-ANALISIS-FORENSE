En la era digital actual, la capacidad de llevar a cabo análisis forenses en ordenadores se ha convertido en una competencia crítica dentro del ámbito de la investigación criminal. El análisis forense informático permite a los investigadores descubrir, preservar y analizar datos en dispositivos electrónicos que pueden ser críticos para resolver delitos. Este capítulo se dedica al estudio meticuloso de los métodos y prácticas estándar en la computación forense, con un enfoque específico en la adquisición y análisis de datos de la memoria RAM y discos duros. Se expondrá la metodología utilizada para garantizar la integridad de la evidencia y se ilustrarán los desafíos asociados a la recolección y el análisis de datos digitales.

Con el avance de la tecnología, los investigadores forenses enfrentan la dualidad de oportunidades y desafíos. Por un lado, las herramientas modernas ofrecen capacidades sin precedentes para recuperar y analizar datos; por otro lado, la creciente sofisticación del software y hardware supone nuevos niveles de complejidad y la necesidad de constante actualización en conocimientos y técnicas. Este capítulo también contempla la noción de deuda técnica asociada a la utilización de herramientas y sistemas operativos en la investigación forense, reconociendo la importancia de mantener un enfoque crítico hacia las herramientas utilizadas.

La documentación y control de versiones son aspectos cruciales en cualquier proyecto de investigación y desarrollo, más aún en el ámbito forense digital, donde la transparencia y reproducibilidad son fundamentales. Se detallará el uso del repositorio de Github (https://github.com/jrodg85/TFM-ANALISIS-FORENSE) para la documentación del TFM y el control de versiones aplicado al proceso de análisis forense. Se discutirá la relevancia de la colaboración y el seguimiento preciso de cambios en el código y documentos relacionados con el proyecto.

Finalmente, no se puede ignorar el papel fundamental que juega el acceso a recursos online en la actualización constante y el acceso a información relevante y actualizada en el campo de la forense digital. La Internet es una fuente inagotable de conocimiento, pero también presenta riesgos que deben ser gestionados con prudencia. En resumen, este capítulo traza el panorama del análisis forense en ordenadores, describiendo las herramientas y metodologías utilizadas, así como las mejores prácticas en la documentación y gestión de la información digital en investigaciones forenses.

Esta introducción proporciona una vista general y establece las expectativas para el contenido que seguirá, preparando al lector para los detalles técnicos y metodológicos que se presentarán en el capítulo.