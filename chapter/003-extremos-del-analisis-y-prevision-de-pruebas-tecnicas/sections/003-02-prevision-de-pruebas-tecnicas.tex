\textbf{Pruebas técnicas:}

\begin{itemize}
    \item El propósito de estas pruebas técnicas es lo indicado en el apartado de problema a resolver del presente Trabajo de fin de master
    \begin{itemize}
        \item Solventar las necesidades del gerente de la empresa mediante el análisis forense del disco duro y la captura de memoria de un ordenador personal, en un caso real con un sistema virtualizado.
        \item Posible vinculación con una presunta conducta delictiva real.
    \end{itemize}
    \item Importancia de las pruebas para validar la hipótesis y objetivos de investigación.
    \begin{itemize}
        \item La posible imputación de los hechos ocurridos y tomar posibles medidas legales contra el autor univoco de la acción detectada.
    \end{itemize}
\end{itemize}

\textbf{Marco metodológico de las pruebas:}

\begin{itemize}
    \item Las pruebas que se realizarán serán una investigación y un estudio temporal de los hechos ocurridos dentro del pc.
    \item Se emplearán herramientas de análisis forense en sus distintos sistemas operativos (Linux/Windows) para su detección.
    \item se tratará de arrancar el sistema virtualizado para posible carving de la información del disco duro por posible eliminación de pruebas por parte del posible infractor.
    \item La planificación de las pruebas ha quedado detallado en la sección "planificación temporal de las tareas".
\end{itemize}


\textbf{Criterios de éxito de las pruebas:}

\begin{itemize}
    \item Análisis de los incidentes ocurridos con una justificación probatoria del mismo.
    \item Realización de un análisis de seguridad de las vulnerabilidades detectadas y una via de mitigación de los mismos.
\end{itemize}

\textbf{Cronograma de pruebas:}

\begin{itemize}
    \item El cronograma de las pruebas ha quedado detallado en la sección "planificación temporal de las tareas".
    \item Hitos importante, fechas de entrega de las PEC.
\end{itemize}

