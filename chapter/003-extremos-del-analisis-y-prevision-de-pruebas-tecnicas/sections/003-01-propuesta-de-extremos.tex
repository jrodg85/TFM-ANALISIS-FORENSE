La presente investigación tiene como propósito fundamental el establecimiento de un marco metodológico para el análisis forense de ordenadores, específicamente orientado hacia la identificación, recolección y análisis de evidencias digitales que puedan ser presentadas en un entorno judicial. A continuación, se delinean los extremos de esta propuesta:

\textbf{Objeto de Estudio:}

\begin{itemize}
  \item La investigación se centrará exclusivamente en el análisis forense del material facilitado para el desarrollo de la asignatura por parte del profesorado de la asignatura.
  \item Se realizará una breve indicacion sobre la aplicacion utilizada con cada uno de los objetivos del presente TFM.
\end{itemize}


\textbf{Alcance metodológico:}

\begin{itemize}
  \item La validación de la integridad de la evidencia se hará mediante el uso de funciones hash estándar.
  \item Se examinarán las metodologías para el análisis de la memoria volátil y no volátil.
\end{itemize}


\textbf{Limitaciones:}

\begin{itemize}
  \item La validación de la integridad de la evidencia se hará mediante el uso de funciones hash estándar.
  \item Se examinarán las metodologías para el análisis de la memoria volátil y no volátil.
\end{itemize}


\textbf{Exclusiones:}

\begin{itemize}
  \item No se utilizará material de analisis que no sea el proporcionado por la asignatura.
\end{itemize}


