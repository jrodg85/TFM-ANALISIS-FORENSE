El análisis forense de la memoria RAM es un componente crítico en la investigación digital, pues permite a los analistas extraer información valiosa que no persiste una vez que el dispositivo se apaga. Esta volatilidad hace que la memoria RAM sea una fuente de evidencia esencial, especialmente en casos donde los procesos activos y la información en tránsito son relevantes para el caso. El presente capítulo detalla un enfoque metodológico estructurado para examinar de manera exhaustiva el contenido de la memoria RAM capturada de un sistema informático, con el objetivo de identificar y analizar aspectos críticos que contribuyan a la investigación.

Las acciones específicas que se abordarán son las siguientes:

\begin{enumerate}
    \item  \textbf{Comprobación del MD5:}

    Iniciaremos con la verificación de la integridad del volcado de la memoria RAM mediante el cálculo de su suma de verificación MD5. Este paso es fundamental para asegurar que los datos analizados no han sido alterados desde el momento de su adquisición, garantizando así la cadena de custodia digital.

    \item \textbf{Identificación del Sistema Operativo:}

    Aunque ya intuyamos, por el apartado anterior datos básicos del Sistema operativo, es vital determinar la versión y configuración del sistema operativo en uso, ya que esto influirá en la interpretación de los datos y en la selección de las herramientas de análisis adecuadas.

    \item \textbf{Búsqueda de Datos de Interés:}

    Seguiremos con la inspección minuciosa del contenido de la memoria para identificar información potencialmente relevante para el caso. Esto incluye, pero no se limita a, datos residuales de aplicaciones, fragmentos de comunicaciones y elementos que puedan ser reconstruidos para obtener evidencia. Servirá para tener una previsión de por donde dirigir el estudio de todo el análisis forense.


    \item \textbf{Búsqueda de Procesos en Funcionamiento de Interés:}

    Un punto focal de nuestra investigación será el examen de los procesos activos en el momento de la captura de la memoria. Esta inspección nos permitirá comprender mejor el estado del sistema antes del apagado o la hibernación.

    \item \textbf{Análisis y Extracción de Procesos Sospechosos:}

    Finalmente, nos concentraremos en reconstruir y examinar las conexiones de red activas y pasivas. El objetivo es identificar patrones de tráfico inusuales o conexiones que puedan indicar comunicación con servidores de comando y control, exfiltración de datos o cualquier otra actividad que se considere sospechosa.

\end{enumerate}

El resultado de este análisis exhaustivo proporcionará una comprensión detallada de lo que estaba ocurriendo en el sistema en el momento de la captura de la memoria. Esta información es invaluable para formar una imagen completa de los eventos bajo investigación y para establecer hechos concretos que puedan ser presentados como evidencia en un entorno judicial.


